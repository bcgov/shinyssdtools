\documentclass[]{article}
\usepackage{lmodern}
\usepackage{amssymb,amsmath}
\usepackage{ifxetex,ifluatex}
\usepackage{fixltx2e} % provides \textsubscript
\ifnum 0\ifxetex 1\fi\ifluatex 1\fi=0 % if pdftex
  \usepackage[T1]{fontenc}
  \usepackage[utf8]{inputenc}
\else % if luatex or xelatex
  \ifxetex
    \usepackage{mathspec}
  \else
    \usepackage{fontspec}
  \fi
  \defaultfontfeatures{Ligatures=TeX,Scale=MatchLowercase}
\fi
% use upquote if available, for straight quotes in verbatim environments
\IfFileExists{upquote.sty}{\usepackage{upquote}}{}
% use microtype if available
\IfFileExists{microtype.sty}{%
\usepackage{microtype}
\UseMicrotypeSet[protrusion]{basicmath} % disable protrusion for tt fonts
}{}
\usepackage[margin=1in]{geometry}
\usepackage{hyperref}
\hypersetup{unicode=true,
            pdfborder={0 0 0},
            breaklinks=true}
\urlstyle{same}  % don't use monospace font for urls
\usepackage{graphicx,grffile}
\makeatletter
\def\maxwidth{\ifdim\Gin@nat@width>\linewidth\linewidth\else\Gin@nat@width\fi}
\def\maxheight{\ifdim\Gin@nat@height>\textheight\textheight\else\Gin@nat@height\fi}
\makeatother
% Scale images if necessary, so that they will not overflow the page
% margins by default, and it is still possible to overwrite the defaults
% using explicit options in \includegraphics[width, height, ...]{}
\setkeys{Gin}{width=\maxwidth,height=\maxheight,keepaspectratio}
\IfFileExists{parskip.sty}{%
\usepackage{parskip}
}{% else
\setlength{\parindent}{0pt}
\setlength{\parskip}{6pt plus 2pt minus 1pt}
}
\setlength{\emergencystretch}{3em}  % prevent overfull lines
\providecommand{\tightlist}{%
  \setlength{\itemsep}{0pt}\setlength{\parskip}{0pt}}
\setcounter{secnumdepth}{0}
% Redefines (sub)paragraphs to behave more like sections
\ifx\paragraph\undefined\else
\let\oldparagraph\paragraph
\renewcommand{\paragraph}[1]{\oldparagraph{#1}\mbox{}}
\fi
\ifx\subparagraph\undefined\else
\let\oldsubparagraph\subparagraph
\renewcommand{\subparagraph}[1]{\oldsubparagraph{#1}\mbox{}}
\fi

%%% Use protect on footnotes to avoid problems with footnotes in titles
\let\rmarkdownfootnote\footnote%
\def\footnote{\protect\rmarkdownfootnote}

%%% Change title format to be more compact
\usepackage{titling}

% Create subtitle command for use in maketitle
\newcommand{\subtitle}[1]{
  \posttitle{
    \begin{center}\large#1\end{center}
    }
}

\setlength{\droptitle}{-2em}

  \title{}
    \pretitle{\vspace{\droptitle}}
  \posttitle{}
    \author{}
    \preauthor{}\postauthor{}
    \date{}
    \predate{}\postdate{}
  
\usepackage{booktabs}
\usepackage{longtable}
\usepackage{array}
\usepackage{multirow}
\usepackage{wrapfig}
\usepackage{float}
\usepackage{colortbl}
\usepackage{pdflscape}
\usepackage{tabu}
\usepackage{threeparttable}
\usepackage{threeparttablex}
\usepackage[normalem]{ulem}
\usepackage{makecell}
\usepackage{xcolor}

\begin{document}

{
\setcounter{tocdepth}{2}
\tableofcontents
}
This app \textbf{fits species sensitivity distributions to concentration
data}. The app is built from the R package
\href{https://github.com/bcgov/ssdtools}{ssdtools}, and shares the same
functionality.

\emph{Hint: Find and click the info icons throughout the app to find
more information on a particular input.}\\
~\\

\hypertarget{step-1-provide-data}{%
\subsubsection{Step 1: Provide data}\label{step-1-provide-data}}

\begin{itemize}
\tightlist
\item
  Data should be provided for \textbf{only one chemical} at a time.
\item
  Each species should not have more than one concentration value.
\item
  Data must have \textbf{at least one column} containing \textbf{at
  least 8 distinct, positive, non-missing concentration values}.
\item
  Optionally, \textbf{species and group} columns can be included, which
  are used to label and color plot output, respectively.\\
\item
  Any additional columns are accepted but are not used by any functions.
\end{itemize}

\begin{table}[H]
\centering\begingroup\fontsize{11}{13}\selectfont

\begin{tabular}{r|l|l}
\hline
Concentration & Species & Group\\
\hline
2.1 & Oncorhynchus mykiss & Fish\\
\hline
2.4 & Ictalurus punctatus & Fish\\
\hline
4.1 & Micropterus salmoides & Fish\\
\hline
10.0 & Brachydanio rerio & Fish\\
\hline
15.6 & Carassius auratus & Fish\\
\hline
18.3 & Pimephales promelas & Fish\\
\hline
6.0 & Daphnia magna & Invertebrate\\
\hline
10.0 & Opercularia bimarginata & Invertebrate\\
\hline
\end{tabular}
\endgroup{}
\end{table}

There are three options to provide data to the app:

\begin{enumerate}
\def\labelenumi{\arabic{enumi}.}
\tightlist
\item
  \textbf{Use the demo Boron dataset}.

  \begin{itemize}
  \tightlist
  \item
    Quickly preview the app functionality on a dataset that `works'.
  \end{itemize}
\item
  \textbf{Upload a csv file}.

  \begin{itemize}
  \tightlist
  \item
    Excel file formats are not accepted. If you have an excel file, try
    exporting a worksheet to csv.
  \end{itemize}
\item
  \textbf{Fill out the interactive table}.

  \begin{itemize}
  \tightlist
  \item
    Species and Group columns are optional. Click on a cell to begin
    entering data. Right-click on the table to delete/insert rows or
    columns. Column names cannot be changed.
  \end{itemize}
\end{enumerate}

Finally, preview the data provided in the table on the right hand side
of the tab.\\
~\\

\begin{figure}
\centering
\includegraphics{https://media.giphy.com/media/fjyG7Wbgp1RKV8sS9s/giphy.gif}
\caption{Provide data in tab \texttt{1.\ Data}}
\end{figure}

\hypertarget{step-2-fit-distributions}{%
\subsubsection{Step 2: Fit
distributions}\label{step-2-fit-distributions}}

\begin{enumerate}
\def\labelenumi{\arabic{enumi}.}
\tightlist
\item
  Specify \textbf{which column contains concentration values}. The app
  attempts to guess which column contains concentration values based on
  data column names. This may need to be corrected.
\item
  \textbf{Select or deselect distributions to fit}. The outputs may take
  a moment to update.
\item
  Format the plot using inputs in the sidebar and \textbf{download plot
  and goodness of fit table} as png and csv files, respectively.\\
\end{enumerate}

\begin{figure}
\centering
\includegraphics{https://media.giphy.com/media/yIjxGcFKt0zK2vj38j/giphy.gif}
\caption{Fit distributions in tab \texttt{2.\ Fit}}
\end{figure}

Additional information about the \textbf{goodness of fit table}: The
columns in the goodness of fit table are the distribution (dist), the
Anderson-Darling statistic (ad), the Kolmogorov-Smirnov statistic (ks),
the Cramer-von Mises statistic (cvm), Akaike's Information Criterion
(aic), Akaike's Information Criterion corrected for sample size (aicc),
Bayesian Information Criterion (bic), the AICc difference (delta) and
the AICc based Akaike weight (weight). The prediction is the model
averaged (using aicc) estimate of the fit. The percent hazard
concentration is the concentration of the chemical which is predicted to
affect that percent of the species tested.\\

\hypertarget{step-3-predict-hazard-concentration}{%
\subsubsection{Step 3: Predict hazard
concentration}\label{step-3-predict-hazard-concentration}}

\begin{enumerate}
\def\labelenumi{\arabic{enumi}.}
\tightlist
\item
  Select the \textbf{threshold \% species affected} to calculate
  \textbf{estimated hazard concentration}. This affects the plot (dotted
  line), text displayed below the plot and calculations of confidence
  limits.
\item
  Select the number of \textbf{bootstrap samples used to calculate
  confidence limits}. The recommended number of samples is 10,000,
  although this can take around 3 minutes to process. Select lower
  number of bootstrap samples to reduce processing time.\\
  ~\\
\end{enumerate}

\begin{table}[H]
\centering\begingroup\fontsize{11}{13}\selectfont

\begin{tabular}{r|l}
\hline
Bootstrap Samples & Estimated processing time\\
\hline
10000 & 3 minutes\\
\hline
5000 & 1.5 minutes\\
\hline
1000 & 15 seconds\\
\hline
500 & 10 seconds\\
\hline
\end{tabular}
\endgroup{}
\end{table}

\begin{enumerate}
\def\labelenumi{\arabic{enumi}.}
\setcounter{enumi}{2}
\tightlist
\item
  Since confidence limits take time to calculate, they are not
  calculated automatically; you must press the \texttt{Get\ CL} button.
\item
  \textbf{Format plot} using various inputs in sidebar and
  \textbf{download plot and table} as png and csv file, respectively.\\
\end{enumerate}

\begin{figure}
\centering
\includegraphics{https://media.giphy.com/media/xKb9nQsPFlqTCGzUgF/giphy.gif}
\caption{Get hazard concentration estimates and confidence limits in tab
\texttt{3.\ Predict}}
\end{figure}

\hypertarget{step-4-get-r-code}{%
\subsubsection{Step 4: Get R code}\label{step-4-get-r-code}}

Copy R code to reproduce outputs programmatically. Code is dynamically
generated based on user inputs and functions exectured within the app
(e.g., code for generating confidence limits will appear after `Get CL'
button is clicked).\\

\begin{figure}
\centering
\includegraphics{https://media.giphy.com/media/XIgsL03rRnEfn8nNas/giphy.gif}
\caption{Get R code to reproduce results programmatically in tab
\texttt{R\ Code}}
\end{figure}


\end{document}
